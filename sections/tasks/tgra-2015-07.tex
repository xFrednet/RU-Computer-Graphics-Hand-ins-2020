\subsection{7. (10\%) Vector intersections and reflections (2015, No. 7)}
A line has end points (3,8) and (7,6). 
A particle starts at (4,2) and travels along in the direction (1,3). 

\subsubsection{a) (7\%) In which point does the path of the particle cross the line?}

The line can also be represented as a point and a direction like:

$
    L
=
    \left(\begin{array}{c}
        3\\
        8
    \end{array}\right)
    +
    a
    *
    \left(
        \left(\begin{array}{c}
            7\\
            6
        \end{array}\right)
        -
        \left(\begin{array}{c}
            3\\
            8
        \end{array}\right)
    \right)
=
    \left(\begin{array}{c}
        3\\
        8
    \end{array}\right)
    +
    a
    *
    \left(\begin{array}{c}
        4\\
        -2
    \end{array}\right)
$

The particle position and movement can be described in the same way:

$
    P
=
    \left(\begin{array}{c}
        4\\
        2
    \end{array}\right)
    +
    b
    *
    \left(\begin{array}{c}
        1\\
        3
    \end{array}\right)
$

We can now create a system of equations (Sorry I don't know the right
english name)

$
    \left(\begin{array}{c}
        3\\
        8
    \end{array}\right)
    +
    a
    *
    \left(\begin{array}{c}
        4\\
        -2
    \end{array}\right)
=
    \left(\begin{array}{c}
        4\\
        2
    \end{array}\right)
    +
    b
    *
    \left(\begin{array}{c}
        1\\
        3
    \end{array}\right)
:
    a=\frac{9}{14}, b=\frac{11}{7}
$

From this solution we can safely say that why collide because:
   1. the equation has a solution
   2. a is < 1 and therefor between (3, 8) and (7, 6)
The particle meets the line at: ~(5.57, 6.71)

$
    P\left(
        b
        =
        \frac{11}{7}
    \right)
=
    \left(\begin{array}{c}
        4\\
        2
    \end{array}\right)
    +
    b
    *
    \left(\begin{array}{c}
        1\\
        3
    \end{array}\right)
\approx
    \left(\begin{array}{c}
        5.57\\
        6.71
    \end{array}\right)
$

This is maybe not the way it's done in games but it works here :)

\subsubsection{b) (3\%) If the particle is made to bounce off the line, what will it’s new direction vector be?}

a = $\left(\begin{array}{c}1\\3\end{array}\right)$

r = $ a - 2 \left(\frac{a \circ n}{n \circ n} * n\right)$

with $|n| = 1$:

r = $ a - 2 (a \circ n) * n $

$
\vec{n}
=
\left(\begin{array}{c} 2 \\ 4 \end{array}\right) * \frac{1}{\sqrt{2^2 + 4^2}}
$ Note: ($|n| = 1$)

r = $
\left(\begin{array}{c}1\\3\end{array}\right)
-
2
\left(
    \left(\begin{array}{c}1\\3\end{array}\right)
    \circ
    \left(\begin{array}{c} 2 \\ 4 \end{array}\right) * \frac{1}{\sqrt{20}}
\right)
*
\left(\begin{array}{c} 2 \\ 4 \end{array}\right) * \frac{1}{\sqrt{20}}
$

r = $
\left(\begin{array}{c}1\\3\end{array}\right)
-
2
\left(
    \frac{14}{\sqrt{20}}
\right)
*
\left(\begin{array}{c} 2 \\ 4 \end{array}\right) * \frac{1}{\sqrt{20}}
$

r = $
\left(\begin{array}{c}1\\3\end{array}\right)
-
\frac{28}{\sqrt{20}}
*
\left(\begin{array}{c} 2 \\ 4 \end{array}\right) * \frac{1}{\sqrt{20}}
$

r = $
\left(\begin{array}{c}1\\3\end{array}\right)
-
\frac{28}{20}
*
\left(\begin{array}{c} 2 \\ 4 \end{array}\right)
$

r = $
\left(\begin{array}{c}-1.8\\-2.6\end{array}\right)
$