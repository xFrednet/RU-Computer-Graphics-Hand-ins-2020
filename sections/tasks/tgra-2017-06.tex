\subsection{(10\%) Vector intersections and reflections (2017 No. 6)}
A line has end points (9, 0) and (7, 4).

A particle starts at (3, 1) and travels along in the direction (6, 2).

\subsubsection{(7\%) In which point does the path of the particle cross the line?} 

\begin{tikzpicture}[scale=1.0]
    \draw[step=1.0cm, gray!10, ultra thin](1, -2) grid (11, 5);
    \draw (9, 0) node[label={right:B (9, 0)}]{} -- (7, 4) node[label={right:C (7, 4)}]{};
    \draw[->] (3, 1) node[label={below:A (3, 1)}]{} -- +(3, 1);
    \draw[->] (9, 0) -- +(-2, -1) node[label={right:n}]{};
    
    \filldraw [gray] (9, 0) circle (3pt);
    \filldraw [gray] (7, 4) circle (3pt);
    \filldraw [gray] (3, 1) circle (3pt);
    \filldraw [gray] (7.7143, 2.5714) circle (3pt) node[label={right:$P_{hit}$}]{};

    \draw ($(9, 0)!0.5!(7, 4)$) node[label={right:v}]{};
    \draw (4.5, 1.5) node[label={above:c}]{};
\end{tikzpicture}


$
    v
=
    C - B
=
    \left(\begin{array}{c}
        -2\\
        4
    \end{array}\right)
$

Line: $
    B
=
    \left(\begin{array}{c}
        9\\
        0
    \end{array}\right)
$
$
    n
=
    v\perp
=
    \left(\begin{array}{c}
        -v.y\\
        v.x
    \end{array}\right)
=
    \left(\begin{array}{c}
        -4\\
        -2
    \end{array}\right)
$

Particle: $
    A
=
    \left(\begin{array}{c}
        9\\
        0
    \end{array}\right)
$
$
    c
=
    \left(\begin{array}{c}
        6\\
        2
    \end{array}\right)
$

$
    t_{hit}
=
    \frac{n * (B - A)}{n * c}
=
    \frac{
        \left(\begin{array}{c}
            -4\\
            -2
        \end{array}\right)
        \circ
        \left(\begin{array}{c}
            9 - 3\\
            0 - 1
        \end{array}\right)
    }{
        \left(\begin{array}{c}
            -4\\
            -2
        \end{array}\right)
        \circ
        \left(\begin{array}{c}
            6\\
            2
        \end{array}\right)
    }
=
    \frac{
        -4 * 6 + -2 * -1
    }{
        -4 * 6 + -2 * 2
    }
=
    \frac{-22}{-28}
=
    \frac{11}{14}
$

$
    P_{hit}
=
    A + t_{hit} * c
=
    \left(\begin{array}{c}
        3\\
        1
    \end{array}\right)
    +
    \frac{11}{14}
    *
    \left(\begin{array}{c}
        6\\
        2
    \end{array}\right)
=
    \left(\begin{array}{c}
        7.7143\\
        2.5714
    \end{array}\right)
$

\subsubsection{(3\%) If the particle is made to bounce off the line, what will its new direction vector be?}

$
    r
=
    c - 2 * \frac{(c \circ n)}{n \circ n} * n
=
    \left(\begin{array}{c}
        6\\
        2
    \end{array}\right)
    -
    2
    *
    \frac{-28}{20}
    *
    \left(\begin{array}{c}
        -4\\
        -2
    \end{array}\right)
=
    \left(\begin{array}{c}
        6\\
        2
    \end{array}\right)
    -
    \left(\begin{array}{c}
        11.2\\
        5.6
    \end{array}\right)
=
    \left(\begin{array}{c}
        -5.2\\
        -3.6
    \end{array}\right)
$