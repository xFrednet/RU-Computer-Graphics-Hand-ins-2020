\subsection{(10\%) Window-2-Viewport mapping (2018 No. 5)}
A second line is drawn into the same window as in the previous example (4. Cohen-Sutherland clipping).  This line has the endpoints: 

\begin{itemize}
    \item Window(left, right, bottom, top) = (-16, 16, -9, 9)
    \item P1 = (-5, 7) \& P2 = (12, -2) 
\end{itemize}

\subsubsection{In which pixels on a 1920x1080 viewport (bottom left corner (0, 0)) will the line's endpoints be rendered? }

\begin{itemize}
    \item viewport(left, right, bottom, top) = (0, 1920, 0, 1080)
\end{itemize}

\begin{itemize}
    \item[A:] x scale (width ratio)
    \item[C:] x translation 
    \item[B:] y scale (height ratio)
    \item[D:] y translation 
\end{itemize}

$
    A 
=
    \frac{viewport\_right \:-\: viewport\_left}{window\_right \:-\: window\_left}
=
    \frac{1920 - 0}{16 - (-16)}
=
    60
$

$
    B
=
    \frac{viewport\_top \:-\: viewport\_bottom}{window\_top \:-\: window\_bottom}
=
    \frac{1080 - 0}{9 - (-9)}
=
    60
$

$
    C
=
    viewport\_left - window\_left * A
=
    0 - (-16) * 60
=
    960
$

$
    D
=
    viewport\_bottom - window\_bottom * B
=
    0 - (-9) * 60
=
    540
$

$
    T
=
    \left(\begin{array}{ccc}
        A & 0 & C\\
        0 & B & D\\
        0 & 0 & 1\\
    \end{array}\right)
=
    \left(\begin{array}{ccc}
        60 &  0 & 960 \\
        0  & 60 & 540 \\
        0  &  0 &    1
    \end{array}\right)
$

$
    T * P1
=
    \left(\begin{array}{c}
        -5 * 60 + 960\\
        7 * 60 + 540\\
        1
    \end{array}\right)
=
    \left(\begin{array}{c}
        660\\
        960\\
        1
    \end{array}\right)
$

$
    T * P2
=
    \left(\begin{array}{c}
        12 * 60 + 960\\
        -2 * 60 + 540\\
        1
    \end{array}\right)
=
    \left(\begin{array}{c}
        1680\\
        420\\
        1
    \end{array}\right)
$

\begin{itemize}
    \item P1 will be drawn at (660, 960)
    \item P2 will be drawn at (1680, 420)
\end{itemize}